\documentclass[
hyperref={pdfpagelabels=false}
%,red %, notes=show
,xcolor=table
% ,handout % UNCOMMENT FOR HANDOUT - also uncomment \pgfpagesuselayout
]
{beamer}

%\usecolortheme{beaver}


\setlength {\marginparwidth }{2cm}
\usepackage{todonotes}

\usepackage{bm}

\definecolor{Myred}{rgb}{0.7, 0.2, 0.2}

\usecolortheme[named=Myred]{structure}


\newcommand{\plus}{{\includegraphics[scale=0.01]{plus.png}}}
\newcommand{\minus}{{\includegraphics[scale=0.06]{minus.png}}}

\usepackage{pres}
% \usepackage{mathdots}
%\usepackage{qtree}
\usepackage{pgfpages}
% \pgfpagesuselayout{4 on 1}[a4paper, landscape,border shrink=5mm]
%\pgfpagesuselayout{2 on 1}[a4paper,border shrink=5mm]
\pgfpageslogicalpageoptions{1}{border code=\pgfusepath{stroke}}
\pgfpageslogicalpageoptions{2}{border code=\pgfusepath{stroke}}
\pgfpageslogicalpageoptions{3}{border code=\pgfusepath{stroke}}
\pgfpageslogicalpageoptions{4}{border code=\pgfusepath{stroke}}

\usepackage{picture}
\usepackage{pgfplots}
\usepackage{filecontents}
% \pgfplotsset{compat=1.12}
\usepackage{pdflscape}

\beamertemplatenavigationsymbolsempty

\newcommand{\travail}{\textbf{ICI IL Y A D TRAVAIL!}}

%%%%%%%%%%%%%%%%%%%%%%%%%%%%%
%%%%% PRESENTATION INFO %%%%%
%%%%%%%%%%%%%%%%%%%%%%%%%%%%%
\title[CSC - Intro]{Cryptography - Principles \\ --Cryptographie et Sécurité des Communications--} 
\author[]{Lionel Morel}
\institute[]{Telecommunications - INSA Lyon}
\date{Fall-Winter 2022-23}


%%%%%%%%%%%%%%%%%%%%%%%%%
%%%%%%%% COLORS %%%%%%%%%
%%%%%%%%%%%%%%%%%%%%%%%%%
\definecolor{greenCiti}{RGB}{83,186,89}
\definecolor{darkGreen}{RGB}{60,132,136}
\definecolor{purple}{RGB}{76, 69, 164}
\colorlet{corecolor}{lightgray}
\definecolor{uncorecolor}{RGB}{222,181,182}
\definecolor{lightgray}{rgb}{0.8,0.8,0.8}
\definecolor{lightblue}{RGB}{188,212,244}
\colorlet{socketcolor}{blue!20}

\colorlet{getpcolor}{red}
\colorlet{leetcolor}{darkGreen}

\definecolor{redfixit}{RGB}{188,43,0}
\definecolor{yellowfixit}{RGB}{235,237,62}
\definecolor{bluefixit}{RGB}{7,2,236}
\definecolor{orangefixit}{RGB}{227,118,24}
\definecolor{cyanfixit}{RGB}{1,171,159}
\definecolor{purplefixit}{RGB}{206,92,232}
\definecolor{greenfixit}{RGB}{102,156,52}



\AtBeginSection[]{
    \begin{frame}
    \vfill
    \centering
    \begin{beamercolorbox}[sep=8pt,center,shadow=true,rounded=true]{section page}
        \usebeamerfont{title}%
        {\color{Myred} \Huge \insertsectionhead}\par
    \end{beamercolorbox}
    \vfill
    \end{frame}
}


\usepackage{tcolorbox}
\tcbset{colframe=white,colback=white,nobeforeafter}


\begin{document}

\begin{frame}
  \maketitle
\end{frame}

\section{Context}

\begin{frame}
  \frametitle{Previously}
  \begin{itemize}
  \item Caesar Cipher
  \item One-Time Pads 
  \item Enigma
  \item \textbf{Cryptology = Cryptography + Cryptanalysis}
  \end{itemize}
\end{frame}


\begin{frame}
  \frametitle{Today's objectives}
  \begin{itemize}
  \item Encryption / Decryption (Confidentiality)
  \item Verification (Integrity)
  \item Signature (Authenticity)
  \end{itemize}
\end{frame}


\begin{frame}
  \frametitle{Kerchoffs Principle {\normalsize (in ``La Cryptographie Militaire'' 1883)}}
  \begin{center}
    \includegraphics[width=\textwidth,height=0.5\textheight,keepaspectratio]{Kerckhoffs.pdf}
  \end{center}
  
  \begin{itemize}
  \item The adversary knows the system [Shannon]
  \item $\ne$ Security by Obscurity
  \item Largely accepted in cryptography
  \item Can be more widely applied to InfoSec (Information System Security) in general. 
  \end{itemize}
\end{frame}


\begin{frame}
  \frametitle{Confusion and Diffusion (Shannon, 1949)}

  \begin{block}{Confusion}
    \begin{itemize}
    \item Each bit in the ciphertext should \textbf{depend on several parts of
      the key}
    \item Usually implemented using \textbf{Substitutions}, aka S-Boxes
    \end{itemize}
  \end{block}


  \begin{block}{Diffusion}
    \begin{itemize}
    \item Encryption/decryptions should imply an \textbf{avalanche effet}. \\ Precisely (in the original Shannon description): changing a single bit in the plaintext changes half of the bits in the cipher-text (eg at the block granularity)
    \item Usually implemented using \textbf{Permutations} (P-Boxes)
    \end{itemize}
  \end{block}
  
\end{frame}



\begin{frame}
  \frametitle{Precautions}

  \begin{itemize}
  \item Use recognized libraries (eg OpenSSL), not your own implementation
  \item Prefer open-source implementations (easier to identify bugs and backdoors)\footnote{\url{https://www.theguardian.com/world/2013/sep/05/nsa-how-to-remain-secure-surveil}lance}
  \item In this class, a lot of simplified versions (same on wikipedia)
  \end{itemize}

  
\end{frame}

\section{Symmetric Cryptography}


\begin{frame}
  \frametitle{Symmetric Cryptography - Principles}

  \begin{center}
    \includegraphics[width=\textwidth]{symCrypto-basic.pdf}
  \end{center}

  \begin{itemize}
  \item Encryption, Decryption, Signature and Verification use the same key
  \item Used implementations are quite efficient. 
  \item A key for each pair of communicating entities
  \item[$\Rightarrow$] Rapid explosion in the number of keys
  \end{itemize}
\end{frame}


\begin{frame}
  \frametitle{Symmetric ciphers - Basic Principles}

  \begin{center}
    \includegraphics[width=\textwidth,height=0.6\textheight,keepaspectratio]{fig/subPerm-principle.pdf}
  \end{center}
  
  Built as a network of substitution/permutation functions:
  \begin{itemize}
  \item Substitution: replace $n$ bits by a pre-determined (but moving) table. Must be one-to-one (to allow reversibility of encryption function)
  \item Permutation: exchange bits
  \end{itemize}
 
\end{frame}

\begin{frame}
  \frametitle{Symmetric ciphers - Basic Principles}

  \begin{block}{Block cipher}
    \begin{itemize}
    \item Treat input as fixed-size blocks (between 64 and 128 bits)
    \item[\plus] More secure
    \item[\minus] Requires padding
    \end{itemize}
  \end{block}


  \begin{block}{Stream cipher}
    \begin{itemize}
    \item Treat input one byte at a time
    \item The encryption of one byte depends on the current state of the cipher (hence of its history of encryption), 
    \item[\plus] fast HW implementation
    \item[\minus] Security less guaranteed
    \end{itemize}
  \end{block}
\end{frame}

\begin{frame}
  \frametitle{Symmetric ciphers - Operation Modes}
  \textbf{Electronic Code Book:}
  \begin{itemize}
  \item Message is divided into blocks and each block is encrypted/decrypted separately
  \end{itemize}

  \begin{center}
    \includegraphics[width=\textwidth,height=0.35\textheight,keepaspectratio]{ECB.png}
  \end{center}

  \begin{itemize}
  \item[\minus] Lacks diffusion 
  \end{itemize}

  \begin{center}
    \includegraphics[width=\textwidth,height=0.3\textheight,keepaspectratio]{ECB-limits.png}
  \end{center}
  
\end{frame}



\begin{frame}
  \frametitle{Symmetric ciphers - Operation Modes}
  \textbf{Cipher Block Chaining}
  \begin{itemize}
  \item Initialization Vector to make all cipher message unique
  \end{itemize}


  \begin{center}
    \includegraphics[width=\textwidth,height=0.6\textheight,keepaspectratio]{CBC.png}
  \end{center}
  \begin{itemize}
  \item[\minus] encryption cannot be parallelized
  \end{itemize}
  
\end{frame}


\begin{frame}
  \frametitle{Symmetric ciphers - Operation Modes}
  \textbf{CounTeR}


  \begin{center}
    \includegraphics[width=\textwidth,height=0.6\textheight,keepaspectratio]{CTR.png}
  \end{center}

\end{frame}





\begin{frame}
  \frametitle{Feistel}
  \begin{minipage}{.35\linewidth}
    \includegraphics[width=\textwidth]{Feistel.pdf}
  \end{minipage}
  \begin{minipage}{.55\linewidth}
    \begin{itemize}
    \item block cipher
    \item $r$ rounds
    \item key $k$ is spilt into $r$ subkeys: $(k_0, ..., k_{r-1})$
    \item plaintext = $(L_0, R_0)$
    \item $(L_{i+1}, R_{i+1}) = (R_i, L_i \oplus f_{k_i}(R_i))$
    \item General structure used in all other ciphers
    \end{itemize}
  \end{minipage}
\end{frame}


\begin{frame}
  \frametitle{Symmetric Cryptography - DES}

  \begin{itemize}
  \item Expands Feistel algorithm, by introducing: 
    \begin{itemize}
    \item More permutations
    \item Substitution Boxes (S-Boxes)
    \end{itemize}
  \item Designed (and initially published) in 1975.
  \item Block-cipher
  \end{itemize}

\end{frame}


% \begin{frame}
%   \frametitle{S-Boxes}


%   \textbf{Redondant!!}
  

%   \begin{itemize}
%   \item Obscure the relationship between the key and the ciphertext
%   \item m-bits input $\longrightarrow$ n-bits output
%   \item Implemented as a lookup table (efficient)
%   \end{itemize}

%   \begin{center}
%     \includegraphics[width=\textwidth,keepaspectratio]{fig/example-Sbox.png}
%   \end{center} 
 
% \end{frame}


\begin{frame}
  \frametitle{DES - General Algorithm}
  \begin{center}
    \includegraphics[height=0.8\textheight,keepaspectratio]{fig/DES1.pdf}
  \end{center} 
  
\end{frame}

\begin{frame}
  \frametitle{DES - One Round}
  \begin{center}
    \includegraphics[height=0.8\textheight,keepaspectratio]{fig/DES-OneRound.pdf}
  \end{center} 

\end{frame}


\begin{frame}
  \frametitle{DES - Weaknesses and Attacks}

  \begin{itemize}
  \item Key size in DES was reduced from 128 bits to 56 bits (after discussions with NSA) ``to fit on a single chip''
  \item Practically cracked (brute-forced) in 1997
  \item Most practical attack to date: still brute force (ie trying out all possible key in turn).
  \item Attacks faster than brute-force:
    \begin{itemize}
    \item Differential cryptanalysis: requires $2^{47}$ chosen plaintexts
    \item Linear cryptanalysis: requires $2^{43}$ chosen plaintexts
    \end{itemize}
  \end{itemize}
\end{frame}


\begin{frame}
  \frametitle{Example: Differential Cryptanalysis}
  
  \textbf{Principle:}
  \begin{itemize}
  \item Choose two plaintexts $x$ and $y$ s.t.\footnote{$\oplus$ = ``xor''}: \\
    $y = x \oplus \Delta_x$ 
  \item Compute the corresponding cyphertexts and for each S-Box S: 
    \begin{itemize}
    \item $S(x,k_i)$
    \item $S(y,k_i) = S(x \oplus \Delta_x,k_i)$
    \end{itemize}
  \item Compute difference on S-Boxes:
    \begin{itemize}
    \item $\Delta_y = S(x \oplus \Delta_x,k_i) \oplus S(x,k_i)$
    \end{itemize}
  \item Repeat this for many plaintexts and several key hypothesis $k_i, i \in \{0, n\}$
  \item key $k_j$ that minimizes $\Delta$ is deemed ``most probable''.
  \end{itemize}

  \textbf{Limits: }
  \begin{itemize}
  \item[\minus] In practice requires $2^{47}$ well-chosen plaintext (so that $\Delta_x$ is ``not too big'')
  \item[\minus] Limits: choose the ``right'' plaintexts
  \end{itemize}
\end{frame}


\begin{frame}
  \frametitle{Bonus: why $\oplus$ (xor) is ``difference''? }

  \begin{tabular}{c|c|c}
    $\oplus$ & 0 & 1 \\\hline
    0        & 0 & 1 \\\hline
    1        & 1 & 0 \\
  \end{tabular}

  \bigskip
  
  Which means $x \oplus y = 1 $ iff $x \neq y$
  
\end{frame}

% \begin{frame}
%   \frametitle{Linear Cryptanalysis}

%   \begin{itemize}
%   \item 
%   \end{itemize}
% \end{frame}


\begin{frame}
  \frametitle{3DES}

  \begin{itemize}
  \item Standardized in 1998 to compensate for the weaknesses of DES
  \item DES has a 56-bits key
  \item 3DES \textbf{chains 3 DES together}:
    \begin{itemize}
    \item Encrypt =  Encrypt($k_1$)$\rightarrow$Decrypt($k_2$)$\rightarrow$Encrypt($k_1$)
    \item Decrypt = Decrypt($k_1$)$\rightarrow$Encrypt($k_2$)$\rightarrow$Decrypt($k_2$)
    \item Key: 112 bits ($k_1|k_2$)
    \end{itemize}
  \item Developped in parallel of AES (waiting for AES to be defined)
  \end{itemize}
 
\end{frame}




\begin{frame}
  \frametitle{AES - Advanced Encryption Standard}
  \begin{itemize}
  \item Supersedes DES
  \item Standardized in 2001
  \item NIST-organized competition with 5 finalists:
    \begin{itemize}
    \item IBM proposed MARS
    \item RSA proposed RC6
    \item Serpent by Anderson, Bihman, Knudsen
    \item Twofish by Bruce Schneier et al
    \item Rijndael, by Daemen and Rijmen
    \end{itemize}
  \item Rijndael's was elected by community after a thourough international comparative effort (including NSA, companies, academics), based on security, performance (speed, memory usage).
  \item NB: no-patent allowed (imposed by the NIST)
  \end{itemize}
\end{frame}




\begin{frame}
  \frametitle{AES - Principle\footnote{\url{https://en.wikipedia.org/wiki/Advanced_Encryption_Standard}}}
  
  \begin{itemize}
  \item AES operates on $4\times 4$ array of 16 bytes, called \textbf{the state}
    \[
      \left[ {\begin{array}{cccc}
                b_0 & b_4 & b_8 & b_{12} \\ 
                b_1 & b_5 & b_9 & b_{13} \\ 
                b_2 & b_6 & b_{10} & b_{14} \\
                b_3 & b_7 & b_{11} & b_{15} \\ 
              \end{array} }
          \right]
        \]
      \item Key size specifies the number of transformation rounds to
        convert input plaintext into output ciphertext:
        \begin{itemize}
        \item 10 rounds for 128-bit keys
        \item 12 rounds for 192-bit keys
        \item 14 rounds for 256-bit keys
        \end{itemize}
      \end{itemize}
      
    \end{frame}


\begin{frame}[fragile]
  \frametitle{AES - Algorithm (for 10 rounds)}
\begin{minted}{C}
void AES_Run_secure(void){
  int i;
  addRoundKey();
  for(i = 0; i < 9; i++){
      subBytes();
      shiftRows();
      mixColumns();
      addRoundKey();
  }
  subBytes();
  shiftRows();
  addRoundKey();
}
\end{minted}
\end{frame}

    
\begin{frame}[fragile]
  \frametitle{AES - Initialization}

  \begin{itemize}
  \item \textbf{KeyExpansion} --- round keys are derived from the
    cipher key using the AES key schedule. AES requires a separate
    128-bit round key block for each round plus one more.
  \item Initial State = Input plaintext
  \end{itemize}
\end{frame}



\begin{frame}[fragile]
  \frametitle{AES - Round Key Addition}

  \begin{itemize}
    \item \textbf{AddRoundKey} – each byte of the state is combined with a byte
      of the round key using bitwise xor.
    \end{itemize}
  
  \begin{center}
  \includegraphics[width=\textwidth,height=0.6\textheight,keepaspectratio]{AES-AddRoundKey.png}
\end{center}
  
  
\end{frame}



\begin{frame}
  \frametitle{AES - SubBytes}

  \begin{itemize}
  \item SubBytes = a non-linear substitution step where each byte is
    replaced with another according to a lookup table.
  \item lookup table = S-box
  \end{itemize}

  \only<1>{
    \begin{center}
      \includegraphics[width=\textwidth,height=0.6\textheight,keepaspectratio]{AES-SubBytes.png}
    \end{center}
  }
  \only<2>{
    \begin{center}
      \includegraphics[width=\textwidth,height=0.6\textheight,keepaspectratio]{AES-S-box.png}
    \end{center}
  }
    
\end{frame}


\begin{frame}
  \frametitle{AES - ShiftRows}
  ShiftRows = a transposition step where the last three rows of the state are shifted cyclically a certain number of steps.  


  \begin{center}
    \includegraphics[width=\textwidth,height=0.6\textheight,keepaspectratio]{AES-ShiftRows.png}
  \end{center}
 
\end{frame}


\begin{frame}
  \frametitle{AES - MixColumns}
  \begin{itemize}
  \item MixColumns = a
    linear mixing operation which operates on the columns of the
    state, combining the four bytes in each column.
    AddRoundKey

  \item Together with ShiftRows, MixColumns provides diffusion in the cipher.
  \end{itemize}
  \begin{center}
    \includegraphics[width=\textwidth,height=0.6\textheight,keepaspectratio]{AES-MixColumns.png}
  \end{center}
  
\end{frame}


\begin{frame}[fragile]
  \frametitle{AES - One Round}

  One Round == 
  \begin{minted}{C}
    subBytes();
    shiftRows();
    mixColumns();
    addRoundKey();  
  \end{minted}

  \bigskip
  
  \begin{itemize}
  \item Repeat 9, 11 or 13 rounds
  \item Plus an extra one without the MixColumns
  \end{itemize}
  
\end{frame}


\begin{frame}
  \frametitle{AES - Weaknesses and Attacks}

  \begin{block}{Related-key attacks exists}
    \begin{itemize}
    \item $2^{99.5}$ time and space complexity
    \item btw: age of universe \~{} $2^{70}$
    \item Anyway totally impractical (because keys are well-chosen to
      be independant in crypto-systems)
    \end{itemize}
  \end{block}

  
  \begin{block}{Side-channel attacks are practical}
  \begin{itemize}
  \item 6-7 blocks plaintexts needed
  \item[$\Rightarrow$] requires HW protections
  \end{itemize}
\end{block}
\end{frame}



\begin{frame}
  \frametitle{Symmetric Cryptogaphy - Conclusions}

  \begin{itemize}
  \item[\plus] Overall very effecient (linear in the size of data to encrypt)
  \item[\plus] Arithmetic/Logical operations are simple: xor. 
  \item[\minus] Requires a shared key! 
  \item Solutions to this:
    \begin{itemize}
    \item Avoid the need for a common key
    \item Find a way to securely share a common key
    \end{itemize}
  \end{itemize}
\end{frame}


% \begin{frame}
%   \frametitle{Aparté}

%   Expliquer qu'il y a des ``secret key storage'' en HW, des trucs inviolables où un fabricant peut installer des clés à l'usine. (mais bon, y a de toute façon la question de la chain of trust). voir \url{https://en.wikipedia.org/wiki/Chain_of_trust}
% \end{frame}



\begin{frame}
  \frametitle{Key Sharing Problem}
  \begin{itemize}
  \item Symmetric cryptography uses same key to encrypt and decrypt
  \item Problem: how to share this key
  \item Hypothesis: there is no secure channel to exchange the key
  \end{itemize}
\end{frame}


\begin{frame}
  \frametitle{Diffie-Hellman Key Exchange}
  \begin{center}
    \includegraphics[width=\textwidth,height=0.8\textheight,keepaspectratio]{fig/DiffieHellman.pdf}
  \end{center}
\end{frame}




\section{Hash}


\begin{frame}
  \frametitle{Cryptographic Hash}


  \begin{center}
    \includegraphics[width=\textwidth]{hmac.pdf}
  \end{center}
  
  \begin{itemize}
  \item eg Hash-based Message Authentication Code
  \item Only sender and recipient can sign/verify the message
  \end{itemize}
  
\end{frame}

\begin{frame}
  \frametitle{Cryptographic Hash - Principle}

  \begin{itemize}
  \item Compute a ``footprint''
  \item The message can be of any size, the footprint is of fixed size
  \item Pseudo-unique identification of message
  \item Used for:
    \begin{itemize}
    \item Integrity checks
    \item Cryptographic signature
    \item PRNG
    \item Hashed password storage
    \end{itemize}
  \end{itemize}
\end{frame}


\begin{frame}
  \frametitle{Cryptographic Hash - Good Properties}

  \begin{itemize}
  \item \textbf{Pre-image resistance}: no one can reverse the hash function (to find input from output)
  \item \textbf{Second pre-image resistance}: unicity of hash. Given an input and the corresponding hash, one cannot find another input with the same hash. 
  \item \textbf{Collision-resistance}: no-one can produce two different inputs with the same hash
  \item \textbf{Randomness}
  \end{itemize}
\end{frame}


\begin{frame}
  \frametitle{Cryptographic Hash - today' state of affairs}

  \begin{block}{Existing (and used) implementations}
    \begin{itemize}
    \item MD5: please don't use anymore: ``cryptographically broken and
      unsuitable for further use''
    \item SHA-1: not recommanded anymore (since 2017)
    \item SHA-2: still not planned for removal
    \item SHA-3: standardized in 2015
    \end{itemize}
  \end{block}
  
  \begin{block}{Current situation}
    \begin{itemize}
    \item Hash functions are critical in crypto!
    \item SHA-2 is still safe but is conceptually close to SHA-1 and might share some weaknesses with it
    \item SHA-3 considered ``as safe'' but built completely differently
    \end{itemize}
  \end{block}
\end{frame}

\section{Asymmetric Cryptography}



\begin{frame}
  \frametitle{(general) Asymmetric Cryptography}

  \begin{itemize}
  \item Each participant $\bm{u}$ has a pair of keys $\bm{(Pub_u,Priv_u)}$. 
  \item $\bm{u}$ sends $\bm{Pub_u}$ to $\bm{v}$
  \item $\bm{v}$ sends $\bm{Pub_v}$ to $\bm{u}$
  \item $\bm{u}$ can encrypt its messages to $\bm{v}$ using a combination of $\bm{Pub_v}$ and $\bm{Priv_u}$
  \item $\bm{v}$ can decrypt messages from $\bm{u}$ using a combination  of $\bm{Pub_u}$ and $\bm{Priv_v}$
  \end{itemize}

  Note:
  \begin{itemize}
  \item Relies on ``hard mathematical problems'':
    \begin{itemize}
    \item Discrete logarithm
    \item Factorization of large numbers
    \end{itemize}
  \item Usually slow (exponentiation)
  \end{itemize}
\end{frame}

\begin{frame}
  \frametitle{RSA}
  \begin{itemize}
  \item Invented in 1977 by Rivest, Shamir and Adleman
  \item MIT Patent in 1983, expired in 2000
  \item Security based on the difficulty of factorizing large integers
  \end{itemize}
\end{frame}


\begin{frame}
  \frametitle{RSA - Key generation}

  \begin{itemize}
  \item Choose $\bm{p}$ and $\bm{q}$, two prime numbers: random, kept secret
  \item Compute $\bm{n = pq}$
  \item Compute $\bm{\lambda(n)}$,
    \begin{itemize}
    \item $\bm{\lambda(n) = lcm(\lambda(p), \lambda(q))}$
    \item $\bm{= lcm(p-1, q-1)}$
    \item $\bm{= \frac{pq}{gcd(p,q)}}$ ... (gcd obtained with Euclid algorithm)
    \end{itemize}
  \item Choose $\bm{e}$ s.t.:
    \begin{itemize}
    \item $\bm{1 < e < \lambda(n)}$
    \item $\bm{gdc(e,\lambda(n))=1}$
    \end{itemize}
  \item Compute $\bm{d = e^{-1} \mbox{ mod } \lambda(n)}$
    \begin{itemize}
    \item $\bm{d}$ is the ``private key exponent''
    \end{itemize}
  \end{itemize}

  \begin{tcolorbox}[colframe=Myred]
    $\bm{Pub = (e,n)}$ \hfill $\bm{Priv = (d,n)}$
  \end{tcolorbox}  
\end{frame}


\begin{frame}
  \frametitle{RSA - Encryption}

  \begin{center}
    \includegraphics[width=\textwidth,height=0.6\textheight,keepaspectratio]{fig/RSA-encryption.pdf}
  \end{center}
  
\end{frame}


\begin{frame}
  \frametitle{RSA - Decryption}

  \begin{center}
    \includegraphics[width=\textwidth,height=0.6\textheight,keepaspectratio]{fig/RSA-decryption.pdf}
  \end{center}
\end{frame}


% \begin{frame}
%   \frametitle{RSA - Signing}

%   \textbf{TODO}

%   (cf slides François)
  
% \end{frame}

\begin{frame}
  \frametitle{RSA - Example}
  \begin{enumerate}
  \item $p = 61$ and $q = 53$
  \item n = pq = 3233
  \item $\lambda(n) = lcm(p-1, q-1)$
  \item $= \lambda(3233) = lcm(60,52) = 780$
  \item Choose $1<e<780$ (coprime to $780$), eg $e=17$
  \item $d = e^{-1} \mbox{ mod } \lambda(n)$
  \item $= 413$ (as $1 = 17 * 413 \mbox{ mod } 780$)
  \item \fbox{Public key = $(e = 17, n = 3233)$}
  \item \fbox{Private key = $(d = 413, m = 3233)$}
  \item $c(m) = m^{17} \mbox { mod } 3233$
  \item $m(c) = c^{413} mod 3233$
  \item $m = 65 \rightarrow c = 65^{17} \mbox{ mod } 3233 = 2790$
  \item $2790 \rightarrow m = 2790^{413} \mbox{ mod } 3233 = 65$
  \end{enumerate}
  
\end{frame}

\begin{frame}
  \frametitle{RSA - Properties \& Limitations}

  \begin{itemize}
  \item[\plus] Finding $d$ requires factorizing $n$ (if finding $p$
    and $q$ s.t. $n=p*q$: proven difficult (for $p$ and $q$ large)
  \item[\minus] Implementation is tricky : good PRNG, acceptable $e$ 
  \item[\minus] Relies on exponentiation which is \textbf{expensive} :
    \begin{center}
      \includegraphics[scale=1]{expon.pdf}
    \end{center}
    \begin{itemize}
    \item Requires a (fast) multiplier
    \item $y$ is big (if you want security)
    \item Way more expensive than xor ! 
    \end{itemize}
  \end{itemize}
\end{frame}


\begin{frame}
  \frametitle{Key management}

  \begin{block}{The key distribution problem}
    \begin{itemize}
    \item To encrypt a message or check a signature, Alice needs Bob's public key
    \item Otherwise, it may encrypt a message thinking only Bob will read it, but maybe Charlie can read it instead
    \item How can she get this public key in a secure manner? 
    \item Hard problem, no perfect solution
    \item[Note: ] Using the right key guarantees Bob \textbf{is} Bob, but not that Bob is honnest ... 
    \end{itemize}
  \end{block}

  \begin{block}{Existing solutions}
    \begin{itemize}
    \item Hierarchical certification authorities
    \item Web of trust (eg PGP)
    \item Direct exchange of keys
    \end{itemize}
  \end{block}
\end{frame}


\section{Hybrid Cryptography}

\begin{frame}
  \frametitle{Comparing Symmetric / Asymmetric cryptogaphy}

  \begin{block}{Symmetric cryptography}
    \begin{itemize}
    \item 1 key per pair of participants ($n^2$ keys)
    \item Fast: simple operations, easy to implement in HW
    \end{itemize}
  \end{block}

  \begin{block}{Asymmetric cryptography}
    \begin{itemize}
    \item 1 pair of key per participant ($2n$ keys)
    \item Slow: complex operations, eg exponentiations
    \end{itemize}
  \end{block}


  \begin{block}{Hybrid cryptography}
    \begin{itemize}
    \item Alice encrypts a symmetric key with the public key of Bob
    \item Alice encrypts the message with the symmetric key
    \item[$\Rightarrow$] Best of both worlds
    \end{itemize}
  \end{block}
  
\end{frame}


\begin{frame}
  \frametitle{The ``best of both worlds''}

  \begin{center}
    \includegraphics[scale=0.25]{best-worlds.png}
  \end{center}

  
  \begin{itemize}
  \item Alice encrypts message with Symm key $k$
  \item Alice encrypts $k$ with Bob's public key
  \item Bob decrypts $k$ with his private key
  \item Bob decrypts message with $k$
  \end{itemize}
  
\end{frame}

\begin{frame}
  \frametitle{Next time}

  \begin{itemize}
  \item Cryptographic protocols
  \item Public Key Authorities
  \item PGP
  \end{itemize}
\end{frame}


\end{document}

  