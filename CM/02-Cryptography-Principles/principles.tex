\documentclass[
hyperref={pdfpagelabels=false}
%,red %, notes=show
,xcolor=table
% ,handout % UNCOMMENT FOR HANDOUT - also uncomment \pgfpagesuselayout
]
{beamer}

%\usecolortheme{beaver}


\setlength {\marginparwidth }{2cm}
\usepackage{todonotes}

\definecolor{Myred}{rgb}{0.7, 0.2, 0.2}

\usecolortheme[named=Myred]{structure}


\newcommand{\plus}{{\includegraphics[scale=0.01]{plus.png}}}
\newcommand{\minus}{{\includegraphics[scale=0.06]{minus.png}}}

\usepackage{pres}
% \usepackage{mathdots}
%\usepackage{qtree}
\usepackage{pgfpages}
% \pgfpagesuselayout{4 on 1}[a4paper, landscape,border shrink=5mm]
%\pgfpagesuselayout{2 on 1}[a4paper,border shrink=5mm]
\pgfpageslogicalpageoptions{1}{border code=\pgfusepath{stroke}}
\pgfpageslogicalpageoptions{2}{border code=\pgfusepath{stroke}}
\pgfpageslogicalpageoptions{3}{border code=\pgfusepath{stroke}}
\pgfpageslogicalpageoptions{4}{border code=\pgfusepath{stroke}}

\usepackage{picture}
\usepackage{pgfplots}
\usepackage{filecontents}
% \pgfplotsset{compat=1.12}
\usepackage{pdflscape}

\beamertemplatenavigationsymbolsempty

\newcommand{\travail}{\textbf{ICI IL Y A D TRAVAIL!}}

%%%%%%%%%%%%%%%%%%%%%%%%%%%%%
%%%%% PRESENTATION INFO %%%%%
%%%%%%%%%%%%%%%%%%%%%%%%%%%%%
\title[CSC - Intro]{Cryptography - Principles \\ --Cryptographie et Sécurité des Communications--} 
\author[]{Lionel Morel}
\institute[]{Telecommunications - INSA Lyon}
\date{Fall-Winter 2021-22}


%%%%%%%%%%%%%%%%%%%%%%%%%
%%%%%%%% COLORS %%%%%%%%%
%%%%%%%%%%%%%%%%%%%%%%%%%
\definecolor{greenCiti}{RGB}{83,186,89}
\definecolor{darkGreen}{RGB}{60,132,136}
\definecolor{purple}{RGB}{76, 69, 164}
\colorlet{corecolor}{lightgray}
\definecolor{uncorecolor}{RGB}{222,181,182}
\definecolor{lightgray}{rgb}{0.8,0.8,0.8}
\definecolor{lightblue}{RGB}{188,212,244}
\colorlet{socketcolor}{blue!20}

\colorlet{getpcolor}{red}
\colorlet{leetcolor}{darkGreen}

\definecolor{redfixit}{RGB}{188,43,0}
\definecolor{yellowfixit}{RGB}{235,237,62}
\definecolor{bluefixit}{RGB}{7,2,236}
\definecolor{orangefixit}{RGB}{227,118,24}
\definecolor{cyanfixit}{RGB}{1,171,159}
\definecolor{purplefixit}{RGB}{206,92,232}
\definecolor{greenfixit}{RGB}{102,156,52}



\AtBeginSection[]{
    \begin{frame}
    \vfill
    \centering
    \begin{beamercolorbox}[sep=8pt,center,shadow=true,rounded=true]{section page}
        \usebeamerfont{title}%
        {\color{Myred} \Huge \insertsectionhead}\par
    \end{beamercolorbox}
    \vfill
    \end{frame}
}



\begin{document}

\begin{frame}
  \maketitle
\end{frame}

\section{Context}

\begin{frame}
  \frametitle{Cryptographie}
\end{frame}


\begin{frame}
  \frametitle{Bases mathématiques}

  \begin{itemize}
  \item Il me faut le modulo
  \item L'exponentiation
  \item 
  \end{itemize}
\end{frame}


\begin{frame}
  \frametitle{One-Time Pads}

  \begin{itemize}
  \item On se souvient de comment ça marche.
  \item On se souvient des limites principales:
    \begin{itemize}
    \item la clé (l'OTP) doit être aussi grande que le message.
    \item Générer du ``Vrai'' random c'est dur. 
    \end{itemize}
  \item On fait remarquer qu'une force des OTPs qu'on veut
    ``retrouver'' (même si un peu moins forte) c'est que le chiffré ne
    doit pas faire ressortir de ``régularité statistique'' du
    plaintext.
  \end{itemize}
  
\end{frame}


\begin{frame}
  \frametitle{Cryptography - Desired Properties}

  Dans le chapitre 6 du live de Barakat, il set dit:

  Two main concepts that build the basis for modern symmetric block
  ciphers are confusion and diffusion.
  \begin{itemize}
  \item Confusion is the concept that a digit resp. bit of the
    ciphertext $c \in C$ depends on several parts of the key $e \in K$.
  \item Diffusion is the concept that if we change a digit resp. bit
    of the plaintext $p\in P$ then the impact on the ciphertext $c \in C$ in 
    unpredictable.
  \end{itemize}

  
\end{frame}


\section{Symmetric Encryption}


\begin{frame}
  \frametitle{Symmetric Cryptography - Principles}

  \begin{center}
    \includegraphics[width=\textwidth]{symCrypto-basic.pdf}
  \end{center}

  \begin{itemize}
  \item Encryption and Decryption use the same key
  \item Used implementations are quit efficient. 
  \item TODO: donner des ordres de grandeur. 
  \end{itemize}
\end{frame}

\begin{frame}
  \frametitle{TODO}

  Stream-cipher vs Block-cipher
\end{frame}


\begin{frame}
  \frametitle{Feistel}
  \begin{minipage}{.35\linewidth}
    \includegraphics[width=\textwidth]{Feistel.pdf}
  \end{minipage}
  \begin{minipage}{.55\linewidth}
    \begin{itemize}
    \item block cipher
    \item $r$ rounds
    \item key $k$ is spilt into $r$ subkeys: $(k_0, ..., k_{r-1})$
    \item plaintext = $(L_0, R_0)$
    \item $(L_{i+1}, R_{i+1}) = (R_i, L_i \oplus f_{k_i}(R_i))$
    \end{itemize}
  \end{minipage}
\end{frame}


\begin{frame}
  \frametitle{Feistel - Weaknesses and Attacks}

  \textbf{TODO}
  
\end{frame}

\begin{frame}
  \frametitle{Symmetric Cryptography - DES}

  Expands Feistel algorithm, by introducing: 
  \begin{itemize}
  \item More permutations
  \item Substitution Boxes (S-Boxes)
  \end{itemize}

  
  \begin{itemize}
  \item Designed (and initially published) in 1975.
  \item Block-cipher
  \end{itemize}

\end{frame}


\begin{frame}
  \frametitle{S-Boxes}

  \begin{itemize}
  \item Obscure the relationship between the key and the ciphertext
  \item m-bits input $\longrightarrow$ n-bits output
  \item Implemented as a lookup table (efficient)
  \end{itemize}

  \begin{center}
    \includegraphics[width=\textwidth,keepaspectratio]{fig/example-Sbox.png}
  \end{center} 
 
\end{frame}


\begin{frame}
  \frametitle{DES - General Algorithm}
  \begin{center}
    \includegraphics[height=0.8\textheight,keepaspectratio]{fig/DES1.pdf}
  \end{center} 
  
\end{frame}

\begin{frame}
  \frametitle{DES - One Round}
  \begin{center}
    \includegraphics[height=0.8\textheight,keepaspectratio]{fig/DES-OneRound.pdf}
  \end{center} 

\end{frame}


\begin{frame}
  \frametitle{DES - Weaknesses and Attacks}

  \begin{itemize}
  \item Most practical attack to date: still brute force (ie trying out all possible key in turn).
  \item Key size in DES was reduced from 128 bits to 56 bits (after discussions with ... NSA) ``to fit on a single chip''
  \item Practically cracked (brute-forced) in 1997
  \item Attacks faster than brute-force:
    \begin{itemize}
    \item Differential cryptanalysis: requires $2^{47}$ chosen plaintexts
    \item Linear cryptanalysis: requires $2^{43}$ chosen plaintexts
    \item TODO: explain each. 
    \end{itemize}
  \end{itemize}
  
\end{frame}



\begin{frame}
  \frametitle{AES - Advanced Encryption Standard}
  \begin{itemize}
  \item Supersedes DES
  \end{itemize}
\end{frame}

\begin{frame}[fragile]
  \frametitle{Algorithm}
\begin{minted}{C}
void AES_Run_secure(void){
  int i;
  addRoundKey();
  for(i = 0; i < 9; i++){
      subBytes();
      shiftRows();
      mixColumns();
      computeKey(rcon[i]);
      addRoundKey();
  }
  subBytes();
  shiftRows();
  computeKey(rcon[i]);
  addRoundKey();
}
\end{minted}
\end{frame}

\begin{frame}[fragile]
  \frametitle{AES explained\footnote{\url{https://en.wikipedia.org/wiki/Advanced_Encryption_Standard}}}

  \begin{itemize}
  \item \textbf{KeyExpansion} --- round keys are derived from the
    cipher key using the AES key schedule. AES requires a separate
    128-bit round key block for each round plus one more.
  \end{itemize}
\end{frame}

\begin{frame}[fragile]
  \frametitle{AES (cont'd)}
  \begin{itemize}
  \item \textbf{Initial round key addition}:

    \begin{itemize}
    \item \textbf{AddRoundKey} – each byte of the state is combined with a byte
      of the round key using bitwise xor.
    \end{itemize}
  \end{itemize}

  \begin{center}
  \includegraphics[width=\textwidth,height=0.6\textheight,keepaspectratio]{AES-AddRoundKey.png}
\end{center}
  
  
\end{frame}



\begin{frame}
  \frametitle{AES - SubBytes}

  SubBytes = a non-linear substitution step where each byte is
  replaced with another according to a lookup table.


  \begin{center}
    \includegraphics[width=\textwidth,height=0.6\textheight,keepaspectratio]{AES-SubBytes.png}
  \end{center}
 
\end{frame}


\begin{frame}
  \frametitle{ShiftRows}
  ShiftRows = a transposition step where the last three rows of the state are shifted cyclically a certain number of steps.  


  \begin{center}
    \includegraphics[width=\textwidth,height=0.6\textheight,keepaspectratio]{AES-ShiftRows.png}
  \end{center}
 
\end{frame}


\begin{frame}
  \frametitle{MixColumns}
  MixColumns = a
      linear mixing operation which operates on the columns of the
      state, combining the four bytes in each column.
      AddRoundKey


  \begin{center}
    \includegraphics[width=\textwidth,height=0.6\textheight,keepaspectratio]{AES-MixColumns.png}
  \end{center}
 
\end{frame}

\begin{frame}
  \frametitle{AES - Weaknesses and Attacks}

  TODO
  
\end{frame}



\begin{frame}
  \frametitle{Symmetric Cryptogaphy - Conclusions}

  \begin{itemize}
  \item[\plus] Overall very effecient (linear in the size of data to encrypt)
  \item[\plus] Arithmetic/Logical operations are simple: xor. 
  \item[\minus] Requires a shared key! 
  \item Solutions to this:
    \begin{itemize}
    \item Secret channel
    \item Avoid the need for a common key
    \item Find a way to securely share a common key
    \end{itemize}
  \end{itemize}
\end{frame}


\begin{frame}
  \frametitle{Aparté}

  Expliquer qu'il y a des ``secret key storage'' en HW, des trucs inviolables où un fabricant peut installer des clés à l'usine. (mais bon, y a de toute façon la question de la chain of trust). voir \url{https://en.wikipedia.org/wiki/Chain_of_trust}
\end{frame}



\begin{frame}
  \frametitle{Key Sharing Problem}
  
\end{frame}


\begin{frame}
  \frametitle{Diffie-Hellman Key Exchange}
  \begin{center}
    \includegraphics[width=\textwidth,height=0.7\textheight,keepaspectratio]{fig/DiffieHellman.pdf}
  \end{center}
\end{frame}


\section{Asymmetric Cryptography}



\begin{frame}
  \frametitle{(general) Asymmetric Cryptography}

  \begin{itemize}
  \item Each participant $u$ has a pair of keys $Pub_u$ and $Priv_u$. 
  \item $u$ sends $Pub_u$ to $v$
  \item $v$ sends $Pub_v$ to $u$
  \item $u$ can encrypt its messages to $v$ using a combination of $Pub_v$ and $Priv_u$
  \item $v$ can decrypt messages from $u$ using a combination  of $Pub_u$ and $Priv_v$
  \end{itemize}
\end{frame}

\begin{frame}
  \frametitle{Rivest–Shamir–Adleman}
\end{frame}



\begin{frame}
  \frametitle{Limits: DH, RSA and Beyond}

  \begin{itemize}
  \item ElGamal ? 
  \end{itemize}
\end{frame}


\end{document}

  