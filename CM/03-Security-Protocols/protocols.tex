\documentclass[
hyperref={pdfpagelabels=false}
%,red %, notes=show
,xcolor=table
% ,handout % UNCOMMENT FOR HANDOUT - also uncomment \pgfpagesuselayout
]
{beamer}

%\usecolortheme{beaver}


\setlength {\marginparwidth }{2cm}
\usepackage{todonotes}

\definecolor{Myred}{rgb}{0.7, 0.2, 0.2}

\usecolortheme[named=Myred]{structure}


\usepackage{pres}
% \usepackage{mathdots}
%\usepackage{qtree}
\usepackage{pgfpages}
% \pgfpagesuselayout{4 on 1}[a4paper, landscape,border shrink=5mm]
%\pgfpagesuselayout{2 on 1}[a4paper,border shrink=5mm]
\pgfpageslogicalpageoptions{1}{border code=\pgfusepath{stroke}}
\pgfpageslogicalpageoptions{2}{border code=\pgfusepath{stroke}}
\pgfpageslogicalpageoptions{3}{border code=\pgfusepath{stroke}}
\pgfpageslogicalpageoptions{4}{border code=\pgfusepath{stroke}}

\usepackage{picture}
\usepackage{pgfplots}
\usepackage{filecontents}
% \pgfplotsset{compat=1.12}
\usepackage{pdflscape}

\beamertemplatenavigationsymbolsempty

\newcommand{\travail}{\textbf{ICI IL Y A D TRAVAIL!}}

%%%%%%%%%%%%%%%%%%%%%%%%%%%%%
%%%%% PRESENTATION INFO %%%%%
%%%%%%%%%%%%%%%%%%%%%%%%%%%%%
\title[CSC - Intro]{Secured Communication Protocols \\ --Cryptographie et Sécurité des Communications--} 
\author[]{Lionel Morel}
\institute[]{Telecommunications - INSA Lyon}
\date{Fall-Winter 2021-22}


%%%%%%%%%%%%%%%%%%%%%%%%%
%%%%%%%% COLORS %%%%%%%%%
%%%%%%%%%%%%%%%%%%%%%%%%%
\definecolor{greenCiti}{RGB}{83,186,89}
\definecolor{darkGreen}{RGB}{60,132,136}
\definecolor{purple}{RGB}{76, 69, 164}
\colorlet{corecolor}{lightgray}
\definecolor{uncorecolor}{RGB}{222,181,182}
\definecolor{lightgray}{rgb}{0.8,0.8,0.8}
\definecolor{lightblue}{RGB}{188,212,244}
\colorlet{socketcolor}{blue!20}

\colorlet{getpcolor}{red}
\colorlet{leetcolor}{darkGreen}

\definecolor{redfixit}{RGB}{188,43,0}
\definecolor{yellowfixit}{RGB}{235,237,62}
\definecolor{bluefixit}{RGB}{7,2,236}
\definecolor{orangefixit}{RGB}{227,118,24}
\definecolor{cyanfixit}{RGB}{1,171,159}
\definecolor{purplefixit}{RGB}{206,92,232}
\definecolor{greenfixit}{RGB}{102,156,52}



\begin{document}

\begin{frame}
  \maketitle
\end{frame}

\section{Context}

\begin{frame}
  \frametitle{}
  \begin{itemize}
  \item Bon jusque là on a juste parlé algo de chiffrement symmétrique
    et assymétrique.
  \item Mais on voit que ça suffit pas, il faut une infrastructure autour de AES+RSA
  \end{itemize}
\end{frame}

\begin{frame}
  \frametitle{Quoi}
  Il s'agît de régler les questions suivantes
  \begin{itemize}
  \item chiffrement/déchiffrement
  \item partage de clé
  \item authentication (signature)
  \end{itemize}
\end{frame}

\begin{frame}
  \frametitle{SSH + openSSH}

  \begin{itemize}
  \item ssh c'est un SecuredShell, cryptographic network protocol
  
  \item OpenSSH c'est toute une infrastructure avec dedans: 
    \begin{itemize}
    \item ssh
    \item scp, sftp
    \item ssh-keygen
    \item a server deamon
  \end{itemize}
  \end{itemize}
\end{frame}



\begin{frame}
  \frametitle{PKI}
\end{frame}


\begin{frame}
  \frametitle{HTTPs}
\end{frame}


\begin{frame}
  \frametitle{}
  La discours général consiste à dire qu'une fois qu'on a DH, RSA et AES on n'a pas tout réglé.
  \begin{itemize}
  \item négociations
  \item 3-rd party certificat storage
  \item PKI centralisé
  \item Double-ratchet (Signal) etc. 
  \item SSL
  \item PGP
  \item SSH
  \end{itemize}
\end{frame}


\begin{frame}
  \frametitle{}
\end{frame}


\begin{frame}
  \frametitle{Hybrid Cryptography}
\end{frame}

\begin{frame}
  \frametitle{Needham-Schroeder}
\end{frame}


\end{document}

  